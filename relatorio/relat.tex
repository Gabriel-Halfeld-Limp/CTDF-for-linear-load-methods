\documentclass[journal]{IEEEtran}
\usepackage{amsmath,amsfonts,amssymb}
\usepackage{algorithmic}
\usepackage{algorithm}
\usepackage{array}
\usepackage[caption=false,font=normalsize,labelfont=sf,textfont=sf]{subfig}
\usepackage{textcomp}
\usepackage{stfloats}
\usepackage{url}
\usepackage{verbatim}
\usepackage{graphicx}
\usepackage{cite}
\usepackage{hyperref}
\hyphenation{op-tical net-works semi-conduc-tor IEEE-Xplore}

\begin{document}

\title{Implementação e Análise dos Fatores de Distribuição de Corrente (CTDF) para Métodos Lineares de Fluxo de Potência}

\author{Gabriel H. Limp, Giovani Junqueira
\thanks{Gabriel H. Limp e Giovani Junqueira são alunos de mestrado em Engenharia Elétrica.}
\thanks{Manuscript submitted June 2025.}
}

\maketitle

\begin{abstract}
Este artigo apresenta a fundamentação matemática e a implementação computacional dos Fatores de Distribuição de Corrente (CTDF) para métodos lineares de fluxo de potência, conforme proposto por Sauer (1981). São detalhadas as formulações dos CTDFs ground, slack-referenced e z-tie, bem como sua implementação em Python. Resultados experimentais são apresentados para sistemas IEEE 14 barras, Sauer 6 e Sauer 11, comparando os métodos AC, DC e CTDF.
\end{abstract}

\begin{IEEEkeywords}
CTDF, Fluxo de Potência, IEEE 14 barras, Python, Sistemas Elétricos, Análise de Redes, Sauer.
\end{IEEEkeywords}

\section{Introdução}
A análise de fluxo de potência é fundamental para o planejamento e operação de sistemas elétricos. Métodos lineares, como o DC Load Flow, são amplamente utilizados devido à sua simplicidade e eficiência. Os Fatores de Distribuição de Corrente (CTDFs) permitem calcular rapidamente o impacto de perturbações de carga ou geração sobre os fluxos de linha, sem a necessidade de resolver novamente todo o sistema. Este trabalho apresenta a fundamentação matemática dos CTDFs, conforme Sauer~\cite{Sauer1981}, e sua implementação computacional em Python.

\section{Fundamentação Matemática dos CTDFs}

\subsection{Modelo Linearizado de Fluxo de Potência (DC Load Flow)}
Considerando um sistema de potência com $n$ barras e $m$ linhas, o fluxo de potência ativa entre as barras $i$ e $j$ é dado por:
\begin{equation}
    P_{ij} = \frac{1}{x_{ij}} (\theta_i - \theta_j)
\end{equation}
onde $x_{ij}$ é a reatância da linha $(i,j)$ e $\theta_i$ é o ângulo da barra $i$.

A equação nodal para todas as barras (exceto a slack) pode ser escrita como:
\begin{equation}
    \mathbf{P} = \mathbf{B} \boldsymbol{\theta}
\end{equation}
onde $\mathbf{P}$ é o vetor de injeções de potência, $\mathbf{B}$ é a matriz de susceptância reduzida, e $\boldsymbol{\theta}$ é o vetor de ângulos.

\subsection{CTDFs: Definição Geral}
O CTDF para uma linha $k$ e barra $m$ é definido como:
\begin{equation}
    CTDF_{k,m} = \frac{\partial f_k}{\partial P_m}
\end{equation}
onde $f_k$ é o fluxo na linha $k$ e $P_m$ é a potência injetada na barra $m$.

\subsection{CTDF Ground-Referenced}
No caso \textbf{ground-referenced}, considera-se que todas as barras estão referenciadas a um "terra" comum (sem slack explícito). O fluxo na linha $k$ entre barras $i$ e $j$ é:
\begin{equation}
    f_k = \frac{1}{x_k} (\theta_i - \theta_j)
\end{equation}
Com $\boldsymbol{\theta} = \mathbf{B}^{-1} \mathbf{P}$, temos:
\begin{equation}
    CTDF^{(g)}_{k,m} = \frac{1}{x_k} \left( e_i^T \mathbf{B}^{-1} e_m - e_j^T \mathbf{B}^{-1} e_m \right)
\end{equation}
onde $e_i$ é o vetor canônico com 1 na posição $i$.

\subsection{CTDF Slack-Referenced}
No caso \textbf{slack-referenced}, uma barra de referência (slack) é explicitamente definida. A matriz $\mathbf{B}$ é reduzida, removendo a linha e coluna da slack. O CTDF é dado por:
\begin{equation}
    CTDF^{(s)}_{k,m} = \frac{1}{x_k} \left( e_i^T \mathbf{B}_{red}^{-1} e_m - e_j^T \mathbf{B}_{red}^{-1} e_m \right)
\end{equation}
onde $m$ e $i,j$ não incluem a slack.

\subsection{CTDF Z-Tie Referenced}
No caso \textbf{z-tie referenced}, considera-se uma ligação fictícia de impedância $z_{tie}$ entre a slack e o "terra". A matriz admitância é modificada para incluir $z_{tie}$, e o CTDF é:
\begin{equation}
    CTDF^{(z)}_{k,m} = \frac{1}{x_k} \left( e_i^T \mathbf{B}_{tie}^{-1} e_m - e_j^T \mathbf{B}_{tie}^{-1} e_m \right)
\end{equation}
onde $\mathbf{B}_{tie}$ é a matriz $\mathbf{B}$ modificada.

\subsection{Resumo Matricial}
De forma compacta, para todas as linhas e barras:
\begin{equation}
    \mathbf{CTDF} = \mathbf{A} \mathbf{B}^{-1}
\end{equation}
onde $\mathbf{A}$ é a matriz de incidência das linhas.

\section{Implementação em Python}

A implementação foi realizada de forma modular, utilizando as bibliotecas \texttt{numpy}, \texttt{matplotlib} e \texttt{pandas}. O código está dividido em:

\begin{itemize}
    \item \textbf{Modelos Elétricos}: Definição de classes para barras, linhas, cargas e geradores.
    \item \textbf{Sistemas}: Instanciação dos sistemas IEEE 14, Sauer 6 e Sauer 11.
    \item \textbf{Fluxo de Potência}: Métodos AC e DC.
    \item \textbf{CTDFs}: Funções para cálculo dos três tipos de CTDF.
    \item \textbf{Notebook principal}: Execução dos experimentos e geração dos gráficos.
\end{itemize}

\subsection{Exemplo de Cálculo do CTDF Ground em Python}
\begin{verbatim}
B = network.get_B()
A = network.incidence_matrix()
CTDF_ground = np.dot(A, np.linalg.inv(B))
\end{verbatim}

\subsection{Execução dos Testes}
Os experimentos foram realizados no notebook \texttt{main.ipynb}, com os seguintes passos:
\begin{enumerate}
    \item Comparação dos ângulos de barra (AC vs DC).
    \item Comparação dos fluxos de linha (DC vs CTDFs).
    \item Análise de perturbação de carga (+10\%) e impacto nos fluxos.
\end{enumerate}

\section{Como Executar o Projeto}

\begin{enumerate}
    \item Clone o repositório:
    \begin{verbatim}
    git clone <URL_DO_REPOSITORIO>
    cd CTDF-for-linear-load-methods
    \end{verbatim}
    \item (Opcional) Crie um ambiente virtual:
    \begin{verbatim}
    python -m venv venv
    venv\Scripts\activate  # Windows
    source venv/bin/activate  # Linux/Mac
    \end{verbatim}
    \item Instale as dependências:
    \begin{verbatim}
    pip install -r requirements.txt
    \end{verbatim}
    \item Execute o notebook:
    \begin{verbatim}
    jupyter notebook
    \end{verbatim}
    \item Abra e execute as células do arquivo \texttt{main.ipynb}.
\end{enumerate}

\section{Resultados dos Experimentos}

Os principais resultados são apresentados em gráficos gerados pelo notebook, incluindo:

\begin{itemize}
    \item Diferença dos ângulos de barra entre AC e DC.
    \item Comparação dos fluxos de linha via DC e CTDF (ground, slack, z-tie).
    \item Variação dos fluxos diante de aumento de carga.
\end{itemize}

% \begin{figure}[!t]
% \centering
% \includegraphics[width=2.5in]{fig1}
% \caption{Comparação dos ângulos de barra (AC vs DC) para o sistema IEEE 14 barras.}
% \label{fig_1}
% \end{figure}

% \begin{figure}[!t]
% \centering
% \includegraphics[width=2.5in]{fig2}
% \caption{Comparação dos fluxos de linha via DC e CTDFs.}
% \label{fig_2}
% \end{figure}

Os resultados mostram que os CTDFs fornecem estimativas rápidas e precisas para pequenas perturbações, sendo especialmente úteis para estudos de sensibilidade.

\section{Conclusão}
A implementação dos CTDFs em Python mostrou-se eficiente para análise de redes elétricas, permitindo estudos rápidos de sensibilidade e contingência. O framework desenvolvido é flexível e pode ser expandido para outros sistemas e métodos.

\section*{Agradecimentos}
Agradecemos ao corpo docente do programa de pós-graduação e aos colegas pelo apoio e discussões construtivas.

\bibliographystyle{IEEEtran}
\begin{thebibliography}{1}
\bibitem{Sauer1981}
P. W. Sauer, ``On The Formulation of Power Distribution Factors for Linear Load Flow Methods,'' \emph{IEEE Transactions on Power Apparatus and Systems}, vol. PAS-100, no. 2, pp. 764-770, Feb. 1981.
\end{thebibliography}

\end{document}